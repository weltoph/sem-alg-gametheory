\documentclass{scrartcl}
\usepackage{pgfplots}
\usepackage{tikz}
\usepackage{csquotes}
\usepackage{framed}
\usepackage{todo}
\usepackage{amsmath}
\usepackage[framed]{ntheorem}

\theoremstyle{nonumberplain}
\newframedtheorem{defi}{Definition}
\newframedtheorem{theo}{Theorem}

\newcommand{\tupel}[1]{\left(#1\right)}
\newcommand{\set}[1]{\left\{#1\right\}}

\title{Resource Buying Games}
\author{Christoph Welzel}

\begin{document}
\definecolor{p1}{rgb}{1,0,0}
\definecolor{p2}{rgb}{0,0,1}
\definecolor{both}{rgb}{0.5,0,0.5}
\usetikzlibrary{arrows, cd, positioning, backgrounds, fit}
\tikzset{%
  node/.style={circle, inner sep=1pt, text width=0pt, text height=0pt, fill=black!100},
  edge/.style={thick, draw}
}
  

\maketitle
\section{Introduction}

\section{Congestion Model}
In this section we want to introduce a structure to model resources and demands
of actors on resources. Therefore, we consider a set of players $N$ and a set
of resources $E$. Every player has a personal aim which can be achieved by
different resources. This is modelled for a player $i\in N$ as family
$\mathcal{S}_{i}$ of sets $S_{i}\subseteq E$ of resources. Additionally, we
want to account for different demands of players on resources by fixing a
demand for every player $i$ as $d_{i}\in\mathbb{N}$. Finally, the cost of a
resource depends on the load the players put onto the resource modelled as
individual cost functions $c_{e}:\mathbb{N}\rightarrow \mathbb{N}$ for every
resource $e\in E$. As an example consider different actors that build a shared
network infrastructure. The connections in the network are resources, the
actors are the players which have individual spots they want to connect.
Different actors might need different bandwidth for connections, i.e. their
demands, and the cost for using connections depends on the used bandwidth.
Accordingly we obtain a congestion model
\begin{equation*}
  \mathcal{M} = \tupel{N, E, \mathcal{S} = \times_{i\in N}\mathcal{S}_{i},
  (d_{i})_{i\in N}, (c_{e})_{e\in E}}
\end{equation*}
for which we introduce some additional definitions:
\begin{itemize}
  \item for a profile $S\in\mathcal{S}$ we define
    the demand on resource $e$ for $S$ as
    $\ell(S) = \sum_{i\in N:e\in S_{i}}d_{i}$
  \item we lift $c_{e}$ to the set $\mathcal{S}$ by setting
    $c_{e}(S) = c_{e}(\ell_{e}(S))$ for $S\in\mathcal{S}$
\end{itemize}
\todo{complete list of auxilary definitions}

\section{Matroids}
In this section we present the combinatorial structure of Matroids.
\todo{why matroids?}

We use matroids as a generalisation of the concept of linear independence.
Matroids can be defined as a ground-set and the family of linear
independent sets of this ground-set. This family has to satisfy three
properties, i.e.
\begin{enumerate}
  \item the empty set is linear independent,
  \item all subset of a linear independent set are linear independent and
  \item for two linear independent sets we can find an element of the larger
    set that forms a linear independent set when added to the smaller set.
\end{enumerate}
We use an equivalent definition of matroids which in a way captures the idea of
bases of a vector space by giving the inclusion-wise maximal linear independent
sets (in analogy also called bases).
\begin{defi}
  The structure $\tupel{E, \mathcal{B}}$ with $\mathcal{B}\subseteq 2^{E}$ is
  a matroid if
  \begin{enumerate}
    \item $\mathcal{B}\neq\emptyset$
    \item let $B,B'\in\mathcal{B}$ and for every $x\in B$ there is $y\in B'$
      s. t. $(B\setminus\set{x})\cup\set{y}\in\mathcal{B}$
  \end{enumerate}
\end{defi}
In analogy to vector spaces we consider every element of a basis to open access
to a dimension. We illustrate this concept with the following example: given an
undirected connected graph $G = \tupel{V, E}$ we will argue that
$\tupel{E, \mathcal{S}}$ with $\mathcal{S}$ as edge-sets of spanning trees
in $G$ forms a matroid. In order to meet the requirement that $\mathcal{S}$
must not be empty we restrict $G$ to those graphs with at least two distinct
vertices. Furthermore, given two spanning trees $\tupel{V, S_{1}}$ and
$\tupel{V, S_{2}}$ of $G$ then removing an edge from $S_{1}$ yields a subgraph
of $G$ that is separated into two connected components. Since $S_{2}$ forms a
spanning tree as well there must be an edge connecting these components which
can be used to again obtain a spanning tree\footnote{see Figure
\ref{fig:spanning} for an illustration}.
\begin{figure}
  \begin{center}
    \begin{tikzpicture}
      \node (t) {\begin{tikzpicture}
  \node[node] (A) {};
  \node[below=0.5 of A] (d1) {};
  \node[node, below=0.5 of d1] (B) {};
  \node[right=1 of A] (d2) {};
  \node[node, right=1 of d2] (D) {};
  \node[node] (C) at (d1-|d2) {};
  \node[node] (E) at (B-|D) {};

  \draw[edge, p1] (A) to (B);
  \draw[edge, p1] (A) to (C);
  \draw[edge, p1] (B) to (C);
  \draw[edge, p1] (C) to (D);
  \draw[edge, p1] (C) to (E);
\end{tikzpicture}
};
      \node[below left=of t] {\begin{tikzpicture}
  \node[node] (A) {};
  \node[below=0.5 of A] (d1) {};
  \node[node, below=0.5 of d1] (B) {};
  \node[right=1 of A] (d2) {};
  \node[node, right=1 of d2] (D) {};
  \node[node] (C) at (d1-|d2) {};
  \node[node] (E) at (B-|D) {};

  \draw[edge, dotted, p1] (A) to (B);
  \draw[edge, p1] (A) to (C);
  \draw[edge, p1] (B) to (C);
  \draw[edge, p1] (C) to (D);
  \draw[edge, p1] (C) to (E);
\end{tikzpicture}
};
      \node[below =of t] {\begin{tikzpicture}
  \node[node] (A) {};
  \node[below=0.5 of A] (d1) {};
  \node[node, below=0.5 of d1] (B) {};
  \node[right=1 of A] (d2) {};
  \node[node, right=1 of d2] (D) {};
  \node[node] (C) at (d1-|d2) {};
  \node[node] (E) at (B-|D) {};

  \draw[edge, p1] (A) to (B);
  \draw[edge, p1, dotted] (A) to (C);
  \draw[edge, p1] (B) to (C);
  \draw[edge, p1] (C) to (D);
  \draw[edge, p1] (C) to (E);
\end{tikzpicture}
};
      \node[below right=of t] {\begin{tikzpicture}
  \node[node] (A) {};
  \node[below=0.5 of A] (d1) {};
  \node[node, below=0.5 of d1] (B) {};
  \node[right=1 of A] (d2) {};
  \node[node, right=1 of d2] (D) {};
  \node[node] (C) at (d1-|d2) {};
  \node[node] (E) at (B-|D) {};

  \draw[edge, dotted, p1] (A) to (B);
  \draw[edge, p1] (A) to (C);
  \draw[edge, p1] (B) to (C);
  \draw[edge, p1] (C) to (D);
  \draw[edge, p1] (C) to (E);
\end{tikzpicture}
};
    \end{tikzpicture}
  \end{center}
  \caption{Graph with its spanning trees.}
  \label{fig:spanning}
\end{figure}
\todo[done]{working iteratively towards a basis by prim's algorithm as example
to introduce cuts and contractions and deletions}
Consider the algorithm of Prim which computes spanning trees by iteratively
adding edges which do not close a cycle: every added edge works towards a basis
in $\mathcal{S}$ and opens a dimensionality by connecting one of the graph
nodes. When an edge is added the result of the algorithm can only compute
certain spanning trees, namely those which contain all edges added up to this
point. Fixing certain elements in matroids and therefore restricting the set of
\enquote{useable} bases is captured by the \emph{contraction} operation on
matroids.
\begin{defi}
  For a matroid $\mathcal{M} = \tupel{E, \mathcal{B}}$ and one $e\in E$ we call
  the matroid $\tupel{E\setminus\set{e}, \set{B\subseteq
  (E\setminus\set{e})\middle| B\cup\set{e}\in\mathcal{B}}}$ the
  \emph{contraction} of $e$ in $\mathcal{M}$, denoted by $\mathcal{M}/e$.
\end{defi}
In a similar manner we introduce the \emph{deletion} operation on matroids to
disregard all bases which use a certain element.
\begin{defi}
  For a matroid $\mathcal{M} = \tupel{E, \mathcal{B}}$ and one $e\in E$ we call
  the matroid $\tupel{E\setminus\set{e}, \set{C\subseteq
  (E\setminus\set{e})\middle| B\in\mathcal{B}}}$ the \emph{deletion} of $e$ in
  $\mathcal{M}$, denoted by $\mathcal{M}\setminus e$.
\end{defi}
Note that this deletion operation is not necessarily well defined for all
elements of the ground-set, e.g. deletion of an edge in the spanning tree
matroid which is a bridge yields an empty set of bases and thus no valid
matroid. Analogously, any set of edges that, when removed, disconnect the graph
contains at least one element necessarily present in any basis. We call such
sets cuts\footnote{see Figure \ref{fig:cuts} for an illustration} and define
them as follows.
\begin{defi}
  Let $\mathcal{M} = \tupel{E, \mathcal{B}}$ be a matroid. A \emph{cut}
  $C\subseteq E$ is an inclusion-wise minimal set s.t. $E\setminus C$ does not
  contain any basis in $\mathcal{B}$.
\end{defi}
\begin{figure}
  \begin{center}
    \begin{tikzpicture}
      \node (1) {\begin{tikzpicture}
  \node[node] (A) {};
  \node[below=0.5 of A] (d1) {};
  \node[node, below=0.5 of d1] (B) {};
  \node[right=1 of A] (d2) {};
  \node[node, right=1 of d2] (D) {};
  \node[node] (C) at (d1-|d2) {};
  \node[node] (E) at (B-|D) {};

  \draw[edge, p1, dotted] (A) to (B);
  \draw[edge, p1, dotted] (A) to (C);
  \draw[edge, p1, dotted] (B) to (C);
  \draw[edge, p1, dotted] (C) to (D);
  \draw[edge, p1] (C) to (E);
\end{tikzpicture}
};
      \node[right=of 1] (2) {\begin{tikzpicture}
  \node[node] (A) {};
  \node[below=0.5 of A] (d1) {};
  \node[node, below=0.5 of d1] (B) {};
  \node[right=1 of A] (d2) {};
  \node[node, right=1 of d2] (D) {};
  \node[node] (C) at (d1-|d2) {};
  \node[node] (E) at (B-|D) {};

  \draw[edge, p1, dotted] (A) to (B);
  \draw[edge, p1, dotted] (A) to (C);
  \draw[edge, p1, dotted] (B) to (C);
  \draw[edge, p1, dotted] (C) to (E);
  \draw[edge, p1] (C) to (D);
\end{tikzpicture}
};
      \node[right=of 2] (3) {\begin{tikzpicture}
  \node[node] (A) {};
  \node[below=0.5 of A] (d1) {};
  \node[node, below=0.5 of d1] (B) {};
  \node[right=1 of A] (d2) {};
  \node[node, right=1 of d2] (D) {};
  \node[node] (C) at (d1-|d2) {};
  \node[node] (E) at (B-|D) {};

  \draw[edge, p1] (A) to (B);
  \draw[edge, p1] (A) to (C);
  \draw[edge, p1, dotted] (B) to (C);
  \draw[edge, p1, dotted] (C) to (D);
  \draw[edge, p1, dotted] (C) to (E);
\end{tikzpicture}
};
      \node[below=of 1] (4) {\begin{tikzpicture}
  \node[node] (A) {};
  \node[below=0.5 of A] (d1) {};
  \node[node, below=0.5 of d1] (B) {};
  \node[right=1 of A] (d2) {};
  \node[node, right=1 of d2] (D) {};
  \node[node] (C) at (d1-|d2) {};
  \node[node] (E) at (B-|D) {};

  \draw[edge, p1] (A) to (B);
  \draw[edge, p1, dotted] (A) to (C);
  \draw[edge, p1] (B) to (C);
  \draw[edge, p1, dotted] (C) to (D);
  \draw[edge, p1, dotted] (C) to (E);
\end{tikzpicture}
};
      \node[right=of 4] (5) {\begin{tikzpicture}
  \node[node] (A) {};
  \node[below=0.5 of A] (d1) {};
  \node[node, below=0.5 of d1] (B) {};
  \node[right=1 of A] (d2) {};
  \node[node, right=1 of d2] (D) {};
  \node[node] (C) at (d1-|d2) {};
  \node[node] (E) at (B-|D) {};

  \draw[edge, p1, dotted] (A) to (B);
  \draw[edge, p1] (A) to (C);
  \draw[edge, p1] (B) to (C);
  \draw[edge, p1, dotted] (C) to (D);
  \draw[edge, p1, dotted] (C) to (E);
\end{tikzpicture}
};
    \end{tikzpicture}
  \end{center}
  \caption{Cuts of the spanning tree matroid from the graph presented in Figure
    \ref{fig:spanning}.}
  \label{fig:cuts}
\end{figure}

\section{Unweighted Matroid Games}


\todos

\end{document}
