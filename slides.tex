\documentclass{beamer}
\usepackage{pgfplots}
\usepackage{tikz}
\usepackage{csquotes}
\usepackage{cancel}
\usepackage{stmaryrd}
\usepackage{url}
\usetheme{metropolis}
\title{Resource Buying Games}
\author[C. Welzel]{Christoph Welzel}

\newcommand{\tupel}[1]{\left(#1\right)}
\newcommand{\set}[1]{\left\{#1\right\}}

\DeclareMathOperator{\ex}{ex}

\begin{document}
\definecolor{p1}{rgb}{1,0,0}
\definecolor{p2}{rgb}{0,0,1}
\definecolor{both}{rgb}{0.5,0,0.5}
\usetikzlibrary{arrows, cd, positioning, backgrounds, fit}
\tikzset{%
  node/.style={circle, inner sep=1pt, text width=0pt, text height=0pt, fill=black!100},
  edge/.style={thick, draw}
}
  

\maketitle

\begin{frame}
  \frametitle{Introduction}
    \begin{itemize}
      \item Resources: Nodes/edges in a graph, Machines to schedule jobs on,
        \dots
      \item Players: jointly buy resources to meet individual
        (potentially overlapping) demands
      \item<2-> Every player contributes to buy certain resources
      \item<3-> Payoff for a player depends on what she payed to meet her demands
    \end{itemize}
    \uncover<4->{\begin{definition}[Pure Nash Equilibrium(informal)]
      A \emph{Pure Nash Equilibrium (PNE)} is given if for every player holds
      that, given all other players stick to their strategies, she cannot alter
      her behaviour to achieve a better payoff for herself.
    \end{definition}}
\end{frame}

\section{Basic Definitions}
\begin{frame}
  \frametitle{Definition: Congestion Model}
  \begin{equation*}
    \mathcal{M} = \tupel{N, E, \mathcal{S}, \left(d_{i}\right)_{i\in N},
    \left(c_{e}\right)_{e\in E}}
  \end{equation*}
  \vspace{-1cm}
  \begin{itemize}
    \item $N$: set of players
    \item<2-> $E$: set of resources
    \item<3-> $\mathcal{S} = \times_{i\in N}\mathcal{S}_{i}$ with $\mathcal{S}_{i}$
      set of desired resources $S_{i}\subseteq E$: set of configurations
    \item<4-> $d_{i}$: demand of player $i$ for her resource
      \begin{itemize}
        \item<4-> $d_{i} = 1$ for an unweighted game
        \item<4-> $\ell_{e}(S) = \sum_{i\in N:e\in S_{i}}d_{i}$: load of $e$
          in configuration $S\in\mathcal{S}$
      \end{itemize}
    \item<5-> $c_{e}:\mathbb{N}\rightarrow\mathbb{N}$ with
      $c_{e}(S) = c_{e}(\ell_{e}(S))$: cost for the resource $e$ under the load
      of $S$
  \end{itemize}
\end{frame}

\begin{frame}
  \frametitle{Definition: Weighted Resource Buying Game}
  \begin{equation*}
    \mathcal{M} = \tupel{N, E, \mathcal{S}, \left(d_{i}\right)_{i\in N},
    \left(c_{e}\right)_{e\in E}}
  \end{equation*}
  \begin{equation*}
    \leadsto\mathcal{G} = \tupel{N, \mathcal{S}\times\mathcal{P}, \pi}
  \end{equation*}
  \vspace{-1cm}
  \begin{itemize}
    \item<2-> $\mathcal{P}=\times_{i\in N} P_{i}$ with
      $P_{i} = \mathbb{R}^{|E|}_{+}$: payment of player $i$
    \item<3-> $e\in E$ \emph{is bought under $S\in \mathcal{S}$} if
      $\sum_{i\in N}p_{i}^{e} \geq c_{e}(S)$
    \item<4-> $\pi_{i}(S) =
      \begin{cases}
        \sum_{e\in E}p_{i}^{e} &\textit{if }S_{i}\textit{ is bought}\\
        \infty &\textit{otherwise}
      \end{cases}$: {private cost of player $i$}
  \end{itemize}
\end{frame}

\begin{frame}
  \frametitle{Restricted Setup}
  \begin{itemize}
    \item \emph{unweighted} games
    \item<2-> marginally non-increasing cost-function $c_{e}$:
      \begin{equation*}
        c_{e}(x + \delta) - c_{e}(x) \geq c_{e}(y + \delta) - c_{e}(y)
        \;\;x\leq y, \delta \in \mathbb{N}
      \end{equation*}
      \begin{center}
        \begin{tikzpicture}
          \begin{axis}[
              xlabel=$x$,
              ylabel=$c_{e}(x)$,
              smooth,
              width=5cm,
              height=3cm,
              ticks=none,
              no markers
            ]
              \addplot{ln(x)};
          \end{axis}
        \end{tikzpicture}
      \end{center}
    \item<3-> restrict players demands to a specific mathematical structure:
      Matroids
  \end{itemize}
\end{frame}

\begin{frame}
  \frametitle{Definition: Matroid}
  \begin{columns}
    \column{0.5\textwidth}
    \begin{equation*}
      M = (E, B)
    \end{equation*}
    \vspace{-0.8cm}
    \begin{itemize}
      \item $E$: ground-set
      \item $B$: set of bases
      \item \emph{basis exchange property}:
        \vspace{-1cm}
        \begin{center}
          \begin{align*}
            \forall &X, Y\in B:\\
            &\forall x\in (X\setminus Y)\\
            &\exists y\in (Y\setminus X):\\
            &\;(X\setminus \set{x})\cup\{y\}\in B
          \end{align*}
        \end{center}
    \end{itemize}
    \column{0.5\textwidth}
    \uncover<2->{Spanning trees:}
    \begin{center}
      \uncover<2->{\resizebox{0.8\textwidth}{!}{\begin{tikzpicture}
  \node[node] (A) {};
  \node[below=0.5 of A] (d1) {};
  \node[node, below=0.5 of d1] (B) {};
  \node[right=1 of A] (d2) {};
  \node[node, right=1 of d2] (D) {};
  \node[node] (C) at (d1-|d2) {};
  \node[node] (E) at (B-|D) {};

  \draw[edge, p1] (A) to (B);
  \draw[edge, p1] (A) to (C);
  \draw[edge, p1] (B) to (C);
  \draw[edge, p1] (C) to (D);
  \draw[edge, p1] (C) to (E);
\end{tikzpicture}
}\\[0.5cm]}
      \uncover<3->{\resizebox{0.4\textwidth}{!}{\begin{tikzpicture}
  \node[node] (A) {};
  \node[below=0.5 of A] (d1) {};
  \node[node, below=0.5 of d1] (B) {};
  \node[right=1 of A] (d2) {};
  \node[node, right=1 of d2] (D) {};
  \node[node] (C) at (d1-|d2) {};
  \node[node] (E) at (B-|D) {};

  \draw[edge, dotted, p1] (A) to (B);
  \draw[edge, p1] (A) to (C);
  \draw[edge, p1] (B) to (C);
  \draw[edge, p1] (C) to (D);
  \draw[edge, p1] (C) to (E);
\end{tikzpicture}
}}
      \uncover<4->{\resizebox{0.4\textwidth}{!}{\begin{tikzpicture}
  \node[node] (A) {};
  \node[below=0.5 of A] (d1) {};
  \node[node, below=0.5 of d1] (B) {};
  \node[right=1 of A] (d2) {};
  \node[node, right=1 of d2] (D) {};
  \node[node] (C) at (d1-|d2) {};
  \node[node] (E) at (B-|D) {};

  \draw[edge, p1] (A) to (B);
  \draw[edge, p1, dotted] (A) to (C);
  \draw[edge, p1] (B) to (C);
  \draw[edge, p1] (C) to (D);
  \draw[edge, p1] (C) to (E);
\end{tikzpicture}
}}
      \uncover<5->{\resizebox{0.4\textwidth}{!}{\begin{tikzpicture}
  \node[node] (A) {};
  \node[below=0.5 of A] (d1) {};
  \node[node, below=0.5 of d1] (B) {};
  \node[right=1 of A] (d2) {};
  \node[node, right=1 of d2] (D) {};
  \node[node] (C) at (d1-|d2) {};
  \node[node] (E) at (B-|D) {};

  \draw[edge, p1] (A) to (B);
  \draw[edge, p1] (A) to (C);
  \draw[edge, p1, dotted] (B) to (C);
  \draw[edge, p1] (C) to (D);
  \draw[edge, p1] (C) to (E);
\end{tikzpicture}
}}
    \end{center}
  \end{columns}
\end{frame}

\begin{frame}
  \frametitle{Cuts in Matroids}
  \begin{columns}
    \column{0.7\textwidth}
    $\mathcal{M} = \tupel{E, \mathcal{B}}$
    \begin{itemize}
      \item \emph{Cut}: inclusion-wise minimal $C\subseteq E$ such that
        $E\setminus C$ does not contain a basis of $\mathcal{M}$
      \item[$\leadsto$] $\mathcal{C}\tupel{\mathcal{M}} =
        \set{C\subseteq E\middle| C\text{ is a cut in } \mathcal{M}}$
    \end{itemize}
    \column{0.3\textwidth}
    \resizebox{0.8\textwidth}{!}{\begin{tikzpicture}
  \node[node] (A) {};
  \node[below=0.5 of A] (d1) {};
  \node[node, below=0.5 of d1] (B) {};
  \node[right=1 of A] (d2) {};
  \node[node, right=1 of d2] (D) {};
  \node[node] (C) at (d1-|d2) {};
  \node[node] (E) at (B-|D) {};

  \draw[edge, p1] (A) to (B);
  \draw[edge, p1] (A) to (C);
  \draw[edge, p1] (B) to (C);
  \draw[edge, p1] (C) to (D);
  \draw[edge, p1] (C) to (E);
\end{tikzpicture}
}
  \end{columns}
  \vspace{0.7cm}
  \begin{columns}
    \column{0.33\textwidth}
    \uncover<2->{\resizebox{0.7\textwidth}{!}{\begin{tikzpicture}
  \node[node] (A) {};
  \node[below=0.5 of A] (d1) {};
  \node[node, below=0.5 of d1] (B) {};
  \node[right=1 of A] (d2) {};
  \node[node, right=1 of d2] (D) {};
  \node[node] (C) at (d1-|d2) {};
  \node[node] (E) at (B-|D) {};

  \draw[edge, p1, dotted] (A) to (B);
  \draw[edge, p1, dotted] (A) to (C);
  \draw[edge, p1, dotted] (B) to (C);
  \draw[edge, p1, dotted] (C) to (D);
  \draw[edge, p1] (C) to (E);
\end{tikzpicture}
}\\[0.3cm]}
    \uncover<3->{\resizebox{0.7\textwidth}{!}{\begin{tikzpicture}
  \node[node] (A) {};
  \node[below=0.5 of A] (d1) {};
  \node[node, below=0.5 of d1] (B) {};
  \node[right=1 of A] (d2) {};
  \node[node, right=1 of d2] (D) {};
  \node[node] (C) at (d1-|d2) {};
  \node[node] (E) at (B-|D) {};

  \draw[edge, p1, dotted] (A) to (B);
  \draw[edge, p1, dotted] (A) to (C);
  \draw[edge, p1, dotted] (B) to (C);
  \draw[edge, p1, dotted] (C) to (E);
  \draw[edge, p1] (C) to (D);
\end{tikzpicture}
}}
    \column{0.33\textwidth}
    \uncover<4->{\resizebox{0.7\textwidth}{!}{\begin{tikzpicture}
  \node[node] (A) {};
  \node[below=0.5 of A] (d1) {};
  \node[node, below=0.5 of d1] (B) {};
  \node[right=1 of A] (d2) {};
  \node[node, right=1 of d2] (D) {};
  \node[node] (C) at (d1-|d2) {};
  \node[node] (E) at (B-|D) {};

  \draw[edge, p1] (A) to (B);
  \draw[edge, p1] (A) to (C);
  \draw[edge, p1, dotted] (B) to (C);
  \draw[edge, p1, dotted] (C) to (D);
  \draw[edge, p1, dotted] (C) to (E);
\end{tikzpicture}
}}\\[0.3cm]
    \uncover<5->{\resizebox{0.7\textwidth}{!}{\begin{tikzpicture}
  \node[node] (A) {};
  \node[below=0.5 of A] (d1) {};
  \node[node, below=0.5 of d1] (B) {};
  \node[right=1 of A] (d2) {};
  \node[node, right=1 of d2] (D) {};
  \node[node] (C) at (d1-|d2) {};
  \node[node] (E) at (B-|D) {};

  \draw[edge, p1] (A) to (B);
  \draw[edge, p1, dotted] (A) to (C);
  \draw[edge, p1] (B) to (C);
  \draw[edge, p1, dotted] (C) to (D);
  \draw[edge, p1, dotted] (C) to (E);
\end{tikzpicture}
}}
    \column{0.33\textwidth}
    \uncover<6->{\resizebox{0.7\textwidth}{!}{\begin{tikzpicture}
  \node[node] (A) {};
  \node[below=0.5 of A] (d1) {};
  \node[node, below=0.5 of d1] (B) {};
  \node[right=1 of A] (d2) {};
  \node[node, right=1 of d2] (D) {};
  \node[node] (C) at (d1-|d2) {};
  \node[node] (E) at (B-|D) {};

  \draw[edge, p1, dotted] (A) to (B);
  \draw[edge, p1] (A) to (C);
  \draw[edge, p1] (B) to (C);
  \draw[edge, p1, dotted] (C) to (D);
  \draw[edge, p1, dotted] (C) to (E);
\end{tikzpicture}
}}
  \end{columns}
\end{frame}

\begin{frame}
  \frametitle{Operations on Matroid $\mathcal{M} = \tupel{E, \mathcal{B}}$}
  \begin{columns}
    \column{0.5\textwidth}
    \begin{center}
      Contraction:
    \end{center}
    \vspace{-0.6cm}
    \begin{align*}
      &\mathcal{M}/e = \tupel{E\setminus\set{e}, \mathcal{B}/e}\text{ with}\\
      &\mathcal{B}/e = \set{B\subseteq (E\setminus\set{e})\middle| B\cup\set{e}\in\mathcal{B}}
    \end{align*}
    \column<4->{0.5\textwidth}
    \begin{center}
      Deletion:
    \end{center}
    \vspace{-0.6cm}
    \begin{align*}
      &\mathcal{M}\setminus e = \tupel{E\setminus\set{e},\mathcal{B}\setminus \set{e}}\text{ with}\\
      &\mathcal{B}\setminus e = \set{B\subseteq E\setminus\set{e}\middle|B\in\mathcal{B}}
    \end{align*}
  \end{columns}
  \begin{center}
    \begin{tikzpicture}
      \uncover<2->{\node (G) {\begin{tikzpicture}
  \node[node] (A) {};
  \node[below=0.5 of A] (d1) {};
  \node[node, below=0.5 of d1] (B) {};
  \node[right=1 of A] (d2) {};
  \node[node, right=1 of d2] (D) {};
  \node[node] (C) at (d1-|d2) {};
  \node[node] (E) at (B-|D) {};

  \draw[edge, p1] (A) to (B);
  \draw[edge, p1] (A) to (C);
  \draw[edge, p1] (B) to (C);
  \draw[edge, p1] (C) to (D);
  \draw[edge, p1] (C) to (E);
\end{tikzpicture}
};
      \node[right=1 of G] (leadsto) {\Huge$\leadsto$};}
      \uncover<3>{\node[right=1 of leadsto] (G') {\begin{tikzpicture}
  \node[node] (A) {};
  \node[below=0.5 of A] (d1) {};
  \node[node, below=0.5 of d1] (B) {};
  \node[right=1 of A] (d2) {};
  \node[node, right=1 of d2] (D) {};
  \node[node] (C) at (d1-|d2) {};
  \node[node] (E) at (B-|D) {};
  
  \node[fit=(A) (B), draw, rectangle, rounded corners, black] (new) {};

  \draw[edge, p1] (new) to (C);
  \draw[edge, p1, dotted] (A) to (C);
  \draw[edge, p1, dotted] (B) to (C);
  \draw[edge, p1] (A) to (B);
  \draw[edge, p1] (C) to (D);
  \draw[edge, p1] (C) to (E);
\end{tikzpicture}
};}
      \uncover<5>{\node[right=1 of leadsto] (G'') {\begin{tikzpicture}
  \node[node] (A) {};
  \node[below=0.5 of A] (d1) {};
  \node[node, below=0.5 of d1] (B) {};
  \node[right=1 of A] (d2) {};
  \node[node, right=1 of d2] (D) {};
  \node[node] (C) at (d1-|d2) {};
  \node[node] (E) at (B-|D) {};

  \node[above left =0.1 and 0.1 of d1] (del1) {};
  \node[above right=0.1 and 0.1 of d1] (del2) {};
  \node[below left =0.1 and 0.1 of d1] (del3) {};
  \node[below right=0.1 and 0.1 of d1] (del4) {};

  \draw[edge, p1, dotted] (A) to (B);
  \draw[edge, p1] (A) to (C);
  \draw[edge, p1] (B) to (C);
  \draw[edge, p1] (C) to (D);
  \draw[edge, p1] (C) to (E);

  \draw[thick, black] (del1) to (del4);
  \draw[thick, black] (del2) to (del3);
\end{tikzpicture}
};}
    \end{tikzpicture}
  \end{center}
\end{frame}

\section{Algorithm: PNE in Unweighted Matroid Games}
\subsection{Algorithm}
\begin{frame}
  \frametitle{Algorithm}
  \begin{tabular}{|lll|}
    \hline
    (i) & Init: &\hspace{-0.6cm}\parbox{0.8\textwidth}{
        \begin{itemize}
          \item \parbox{0.7\textwidth}{every player: empty working basis, complete ground-set, complete set of valid result bases}
          \item every resource: cost for next load, current load
        \end{itemize}
      }\\\hline
    (ii) & Find: &\hspace{-0.6cm}\parbox{0.8\textwidth}{
        \begin{itemize}
          \item inclusion-min (necessary cost)-max cut $C$
          \item minimal cost resource $e$ in $C$
          \item owner $i$ of $C$
        \end{itemize}
      }\\\hline
    (iii) & Pick: &\hspace{-0.6cm}\parbox{0.8\textwidth}{
        \begin{itemize}
          \item $i$: add $e$ to working basis, contract matroid to $e$
          \item $e$: update marginal cost and increase load
          \item all player: delete elements $C\setminus\set{e}$ from matroid
        \end{itemize}
      }\\\hline
    (iv) & Break: &\hspace{-0.6cm}\parbox{0.8\textwidth}{
        \begin{itemize}
          \item exists player with incomplete basis: goto (ii)
          \item otherwise: return chosen bases and payments
        \end{itemize}
      }\\\hline
  \end{tabular}
\end{frame}

\begin{frame}
  \frametitle{Example: Graph}
  \begin{columns}
    \column{0.5\textwidth}
    Player 1:\\
    \vspace{-0.2cm}
    \begin{center}
      \resizebox{0.6\textwidth}{!}{\begin{tikzpicture}
  \node[node, label=$A$] (A) {};
  \node[below=0.5 of A] (d1) {};
  \node[node, below=0.5 of d1] (B) {};
  \node[right=1 of A] (d2) {};
  \node[node, right=1 of d2, label=$D$] (D) {};
  \node[node, label=$C$] (C) at (d1-|d2) {};
  \node[node] (E) at (B-|D) {};

  \draw[edge, p1] (A) to (B);
  \draw[edge, p1] (A) to (C);
  \draw[edge, p1] (B) to (C);
  \draw[edge, p1] (C) to (D);
  \draw[edge, p1] (C) to (E);
\end{tikzpicture}
}
    \end{center}
    \column{0.5\textwidth}
    Player 2:\\
    \vspace{-0.5cm}
    \begin{center}
      \resizebox{0.6\textwidth}{!}{\begin{tikzpicture}
  \node[node, label=$A$] (A) {};
  \node[below=0.5 of A] (d1) {};
  \node[right=1 of A] (d2) {};
  \node[node, right=1 of d2, label=$D$] (D) {};
  \node[node, label=$C$] (C) at (d1-|d2) {};
  \node[node, above=0.5 of d2] (F) {};

  \draw[edge, p2] (A) to (C);
  \draw[edge, p2] (C) to (D);
  \draw[edge, p2] (A) to (F);
  \draw[edge, p2] (D) to (F);
\end{tikzpicture}
}
    \end{center}
  \end{columns}
  \begin{center}
    \resizebox{0.35\textwidth}{!}{\begin{tikzpicture}
  \node[node] (A) {};
  \node[below=0.5 of A] (d1) {};
  \node[node, below=0.5 of d1] (B) {};
  \node[right=1 of A] (d2) {};
  \node[node, right=1 of d2] (D) {};
  \node[node] (C) at (d1-|d2) {};
  \node[node] (E) at (B-|D) {};
  \node[node, above=0.5 of d2] (F) {};

  \draw[edge, p1] (A) to (B);
  \draw[edge, both] (A) to (C);
  \draw[edge, p1] (B) to (C);
  \draw[edge, both] (C) to (D);
  \draw[edge, p1] (C) to (E);
  \draw[edge, p2] (A) to (F);
  \draw[edge, p2] (D) to (F);
\end{tikzpicture}
}
  \end{center}
\end{frame}

\newlength{\cutscaling}
\setlength{\cutscaling}{0.17\textwidth}
\begin{frame}
  \frametitle{Example: Algorithm}
  \begin{columns}
    % left column:
    \column{0.3\textwidth}
    \resizebox{\textwidth}{!}{\alt<1-6>{\begin{tikzpicture}
  \node[node] (A) {};
  \node[below=0.5 of A] (d1) {};
  \node[node, below=0.5 of d1] (B) {};
  \node[right=1 of A] (d2) {};
  \node[node, right=1 of d2] (D) {};
  \node[node] (C) at (d1-|d2) {};
  \node[node] (E) at (B-|D) {};

  \draw[edge, p1] (A) to (B);
  \draw[edge, p1] (A) to (C);
  \draw[edge, p1] (B) to (C);
  \draw[edge, p1] (C) to (D);
  \draw[edge, p1] (C) to (E);
\end{tikzpicture}
}{\alt<7>{\begin{tikzpicture}
  \node[node] (A) {};
  \node[below=0.5 of A] (d1) {};
  \node[node, below=0.5 of d1] (B) {};
  \node[right=1 of A] (d2) {};
  \node[node, right=1 of d2] (D) {};
  \node[node] (C) at (d1-|d2) {};
  \node[node] (E) at (B-|D) {};
  
  \node[fit=(A) (B), draw, rectangle, rounded corners, black] (new) {};

  \draw[edge, p1] (new) to (C);
  \draw[edge, p1, dotted] (A) to (C);
  \draw[edge, p1, dotted] (B) to (C);
  \draw[edge, p1] (A) to (B);
  \draw[edge, p1] (C) to (D);
  \draw[edge, p1] (C) to (E);
\end{tikzpicture}
}{\begin{tikzpicture}
  \node[node] (A) {};
  \node[below=0.5 of A] (d1) {};
  \node[node, below=0.5 of d1] (B) {};
  \node[right=1 of A] (d2) {};
  \node[node, right=1 of d2] (D) {};
  \node[node] (C) at (d1-|d2) {};
  \node[node] (E) at (B-|D) {};
  
  \node[fit=(A) (B), draw, rectangle, rounded corners, black] (new) {};

  \draw[edge, p1, bend right, dotted] (C) to (new.east);
  \draw[edge, p1, bend left] (C) to (new.east);
  \draw[edge, p1] (A) to (B);
  \draw[edge, p1] (C) to (D);
  \draw[edge, p1] (C) to (E);
\end{tikzpicture}
}}}
    \begin{columns}
      \column{0.5\textwidth}
      \resizebox{\cutscaling}{!}{\alt<1-2>{\begin{tikzpicture}
  \node[node] (A) {};
  \node[below=0.5 of A] (d1) {};
  \node[node, below=0.5 of d1] (B) {};
  \node[right=1 of A] (d2) {};
  \node[node, right=1 of d2] (D) {};
  \node[node] (C) at (d1-|d2) {};
  \node[node] (E) at (B-|D) {};

  \draw[edge, p1, dotted] (A) to (B);
  \draw[edge, p1, dotted] (A) to (C);
  \draw[edge, p1, dotted] (B) to (C);
  \draw[edge, p1, dotted] (C) to (D);
  \draw[edge, p1] (C) to (E);
\end{tikzpicture}
}{\begin{tikzpicture}
  \begin{scope}[on background layer]
    \node[node] (A) {};
    \node[below=0.5 of A] (d1) {};
    \node[node, below=0.5 of d1] (B) {};
    \node[right=1 of A] (d2) {};
    \node[node, right=1 of d2] (D) {};
    \node[node] (C) at (d1-|d2) {};
    \node[node] (E) at (B-|D) {};


    \draw[edge, p1, dotted] (A) to (B);
    \draw[edge, p1, dotted] (A) to (C);
    \draw[edge, p1, dotted] (B) to (C);
    \draw[edge, p1, dotted] (C) to (D);
    \draw[edge, p1] (C) to (E);
  \end{scope}
  \node[fill=green!25, circle, draw, text=black, inner sep=2pt] (value) at (d1-|d2) {\huge$6$};
\end{tikzpicture}
}}
      \resizebox{\cutscaling}{!}{\alt<1-3>{\begin{tikzpicture}
  \node[node] (A) {};
  \node[below=0.5 of A] (d1) {};
  \node[node, below=0.5 of d1] (B) {};
  \node[right=1 of A] (d2) {};
  \node[node, right=1 of d2] (D) {};
  \node[node] (C) at (d1-|d2) {};
  \node[node] (E) at (B-|D) {};

  \draw[edge, p1, dotted] (A) to (B);
  \draw[edge, p1, dotted] (A) to (C);
  \draw[edge, p1, dotted] (B) to (C);
  \draw[edge, p1, dotted] (C) to (E);
  \draw[edge, p1] (C) to (D);
\end{tikzpicture}
}{\begin{tikzpicture}
  \begin{scope}[on background layer]
    \node[node] (A) {};
    \node[below=0.5 of A] (d1) {};
    \node[node, below=0.5 of d1] (B) {};
    \node[right=1 of A] (d2) {};
    \node[node, right=1 of d2] (D) {};
    \node[node] (C) at (d1-|d2) {};
    \node[node] (E) at (B-|D) {};

    \draw[edge, p1, dotted] (A) to (B);
    \draw[edge, p1, dotted] (A) to (C);
    \draw[edge, p1, dotted] (B) to (C);
    \draw[edge, p1, dotted] (C) to (E);
    \draw[edge, p1] (C) to (D);
  \end{scope}
  \node[fill=green!25, circle, draw, text=black, inner sep=2pt] (value) at (d1-|d2) {\huge$3$};
\end{tikzpicture}
}}
      \resizebox{\cutscaling}{!}{\alt<1-3>{\begin{tikzpicture}
  \node[node] (A) {};
  \node[below=0.5 of A] (d1) {};
  \node[node, below=0.5 of d1] (B) {};
  \node[right=1 of A] (d2) {};
  \node[node, right=1 of d2] (D) {};
  \node[node] (C) at (d1-|d2) {};
  \node[node] (E) at (B-|D) {};

  \draw[edge, p1] (A) to (B);
  \draw[edge, p1] (A) to (C);
  \draw[edge, p1, dotted] (B) to (C);
  \draw[edge, p1, dotted] (C) to (D);
  \draw[edge, p1, dotted] (C) to (E);
\end{tikzpicture}
}{\begin{tikzpicture}
  \begin{scope}[on background layer]
    \node[node] (A) {};
    \node[below=0.5 of A] (d1) {};
    \node[node, below=0.5 of d1] (B) {};
    \node[right=1 of A] (d2) {};
    \node[node, right=1 of d2] (D) {};
    \node[node] (C) at (d1-|d2) {};
    \node[node] (E) at (B-|D) {};

    \draw[edge, p1] (A) to (B);
    \draw[edge, p1] (A) to (C);
    \draw[edge, p1, dotted] (B) to (C);
    \draw[edge, p1, dotted] (C) to (D);
    \draw[edge, p1, dotted] (C) to (E);
  \end{scope}
  \node[fill=green!25, circle, draw, text=black, inner sep=2pt] (value) at (d1-|d2) {\huge$7$};
\end{tikzpicture}
}}
      \column{0.5\textwidth}
      \resizebox{\cutscaling}{!}{\alt<1-3>{\begin{tikzpicture}
  \node[node] (A) {};
  \node[below=0.5 of A] (d1) {};
  \node[node, below=0.5 of d1] (B) {};
  \node[right=1 of A] (d2) {};
  \node[node, right=1 of d2] (D) {};
  \node[node] (C) at (d1-|d2) {};
  \node[node] (E) at (B-|D) {};

  \draw[edge, p1] (A) to (B);
  \draw[edge, p1, dotted] (A) to (C);
  \draw[edge, p1] (B) to (C);
  \draw[edge, p1, dotted] (C) to (D);
  \draw[edge, p1, dotted] (C) to (E);
\end{tikzpicture}
}{\begin{tikzpicture}
  \begin{scope}[on background layer]
    \node[node] (A) {};
    \node[below=0.5 of A] (d1) {};
    \node[node, below=0.5 of d1] (B) {};
    \node[right=1 of A] (d2) {};
    \node[node, right=1 of d2] (D) {};
    \node[node] (C) at (d1-|d2) {};
    \node[node] (E) at (B-|D) {};

    \draw[edge, p1] (A) to (B);
    \draw[edge, p1, dotted] (A) to (C);
    \draw[edge, p1] (B) to (C);
    \draw[edge, p1, dotted] (C) to (D);
    \draw[edge, p1, dotted] (C) to (E);
  \end{scope}
  \node[fill=green!25, circle, draw, text=black, inner sep=2pt] (value) at (d1-|d2) {\huge$2$};
\end{tikzpicture}
}}
      \resizebox{\cutscaling}{!}{\alt<1-3>{\begin{tikzpicture}
  \node[node] (A) {};
  \node[below=0.5 of A] (d1) {};
  \node[node, below=0.5 of d1] (B) {};
  \node[right=1 of A] (d2) {};
  \node[node, right=1 of d2] (D) {};
  \node[node] (C) at (d1-|d2) {};
  \node[node] (E) at (B-|D) {};

  \draw[edge, p1, dotted] (A) to (B);
  \draw[edge, p1] (A) to (C);
  \draw[edge, p1] (B) to (C);
  \draw[edge, p1, dotted] (C) to (D);
  \draw[edge, p1, dotted] (C) to (E);
\end{tikzpicture}
}{\begin{tikzpicture}
  \begin{scope}[on background layer]
    \node[node] (A) {};
    \node[below=0.5 of A] (d1) {};
    \node[node, below=0.5 of d1] (B) {};
    \node[right=1 of A] (d2) {};
    \node[node, right=1 of d2] (D) {};
    \node[node] (C) at (d1-|d2) {};
    \node[node] (E) at (B-|D) {};

    \draw[edge, p1, dotted] (A) to (B);
    \draw[edge, p1] (A) to (C);
    \draw[edge, p1] (B) to (C);
    \draw[edge, p1, dotted] (C) to (D);
    \draw[edge, p1, dotted] (C) to (E);
  \end{scope}
  \node[fill=green!25, circle, draw, text=black, inner sep=2pt] (value) at (d1-|d2) {\huge$2$};
\end{tikzpicture}
}}
    \end{columns}

    % middle column:
    \column{0.4\textwidth}
    \resizebox{\textwidth}{!}{\alt<1-8>{\begin{tikzpicture}
  \node[node] (A) {};
  \node[below=0.5 of A] (d1) {};
  \node[node, below=0.5 of d1] (B) {};
  \node[right=1 of A] (d2) {};
  \node[node, right=1 of d2] (D) {};
  \node[node] (C) at (d1-|d2) {};
  \node[node] (E) at (B-|D) {};
  \node[node, above=0.5 of d2] (F) {};

  \draw[edge, p1] (A) to node[right, black] {\tiny$7$} (B);
  \draw[edge, both] (A) to node[above, black] {\tiny$8$} (C);
  \draw[edge, p1] (B) to node[below, black] {\tiny$2$} (C);
  \draw[edge, both] (C) to node[above, black] {\tiny$3$} (D);
  \draw[edge, p1] (C) to node[below, black] {\tiny$6$} (E);
  \draw[edge, p2] (A) to node[above, black] {\tiny$5$} (F);
  \draw[edge, p2] (D) to node[above, black] {\tiny$2$} (F);
\end{tikzpicture}
}{\begin{tikzpicture}
  \node[node] (A) {};
  \node[below=0.5 of A] (d1) {};
  \node[node, below=0.5 of d1] (B) {};
  \node[right=1 of A] (d2) {};
  \node[node, right=1 of d2] (D) {};
  \node[node] (C) at (d1-|d2) {};
  \node[node] (E) at (B-|D) {};
  \node[node, above=0.5 of d2] (F) {};

  \draw[edge, p1] (A) to node[right, black, label={[black,label distance=-0.2cm]right:\tiny$8$}] {\tiny\bcancel{$7$}} (B);
  \draw[edge, both, dotted] (A) to node[above, black] {\tiny$8$} (C);
  \draw[edge, p1] (B) to node[below, black] {\tiny$2$} (C);
  \draw[edge, both] (C) to node[above, black] {\tiny$3$} (D);
  \draw[edge, p1] (C) to node[below, black] {\tiny$6$} (E);
  \draw[edge, p2] (A) to node[above, black] {\tiny$5$} (F);
  \draw[edge, p2] (D) to node[above, black] {\tiny$2$} (F);
\end{tikzpicture}
}}
    \begin{center}
      \uncover<5->{\alt<5>{\resizebox{0.7\textwidth}{!}{\begin{tikzpicture}
  \node[node] (A) {};
  \node[below=0.5 of A] (d1) {};
  \node[node, below=0.5 of d1] (B) {};
  \node[right=1 of A] (d2) {};
  \node[node, right=1 of d2] (D) {};
  \node[node] (C) at (d1-|d2) {};
  \node[node] (E) at (B-|D) {};

  \draw[edge, p1] (A) to (B);
  \draw[edge, p1] (A) to (C);
  \draw[edge, p1, dotted] (B) to (C);
  \draw[edge, p1, dotted] (C) to (D);
  \draw[edge, p1, dotted] (C) to (E);
\end{tikzpicture}
}}{\resizebox{0.7\textwidth}{!}{\begin{tikzpicture}
  \node[node] (A) {};
  \node[below=0.5 of A] (d1) {};
  \node[node, below=0.5 of d1] (B) {};
  \node[right=1 of A] (d2) {};
  \node[node, right=1 of d2] (D) {};
  \node[node] (C) at (d1-|d2) {};
  \node[node] (E) at (B-|D) {};

  \draw[edge, p1] (A) to node[right, black] {$e$} (B);
  \draw[edge, p1] (A) to (C);
  \draw[edge, p1, dotted] (B) to (C);
  \draw[edge, p1, dotted] (C) to (D);
  \draw[edge, p1, dotted] (C) to (E);
\end{tikzpicture}
}}}
    \end{center}

    % right column:
    \column{0.3\textwidth}
    \resizebox{\textwidth}{!}{\alt<1-7>{\begin{tikzpicture}
  \node[node] (A) {};
  \node[right=1 of A] (d1) {};
  \node[node, right=1 of d1] (C) {};
  \node[above=0.5 of C] (d2) {};
  \node[node] (B) at (d2-|d1) {};
  \node[node, above=0.5 of d2] (D) {};
  \node[node, above=0.2 of D] (E) {};

  \draw[edge, p2] (A) to (B);
  \draw[edge, p2] (B) to (D);
  \draw[edge, p2] (C) to (D);
  \draw[edge, p2] (B) to (E);
  \draw[edge, p2, bend right] (C) to (E);
\end{tikzpicture}
}{\begin{tikzpicture}
  \node[node] (A) {};
  \node[below=0.5 of A] (d1) {};
  \node[right=1 of A] (d2) {};
  \node[node, right=1 of d2] (D) {};
  \node[node] (C) at (d1-|d2) {};
  \node[node, above=0.5 of d2] (F) {};

  \draw[edge, p2, dotted] (A) to (C);
  \draw[edge, p2] (C) to (D);
  \draw[edge, p2] (A) to (F);
  \draw[edge, p2] (D) to (F);
\end{tikzpicture}
}}
    \begin{columns}
      \column{0.5\textwidth}
      \resizebox{\cutscaling}{!}{\alt<1-3>{\begin{tikzpicture}
  \node[node] (A) {};
  \node[below=0.5 of A] (d1) {};
  \node[right=1 of A] (d2) {};
  \node[node, right=1 of d2] (D) {};
  \node[node] (C) at (d1-|d2) {};
  \node[node, above=0.5 of d2] (F) {};

  \draw[edge, p2, dotted] (A) to (C);
  \draw[edge, p2, dotted] (C) to (D);
  \draw[edge, p2] (A) to (F);
  \draw[edge, p2] (D) to (F);
\end{tikzpicture}
}{\begin{tikzpicture}
  \begin{scope}[on background layer]
    \node[node] (A) {};
    \node[below=0.5 of A] (d1) {};
    \node[right=1 of A] (d2) {};
    \node[node, right=1 of d2] (D) {};
    \node[node] (C) at (d1-|d2) {};
    \node[node, above=0.5 of d2] (F) {};

    \draw[edge, p2, dotted] (A) to (C);
    \draw[edge, p2, dotted] (C) to (D);
    \draw[edge, p2] (A) to (F);
    \draw[edge, p2] (D) to (F);
  \end{scope}
  \node[fill=green!25, circle, draw, text=black, inner sep=2pt] (value) at (d2) {\huge$2$};
\end{tikzpicture}
}}
      \resizebox{\cutscaling}{!}{\alt<1-3>{\begin{tikzpicture}
  \node[node] (A) {};
  \node[below=0.5 of A] (d1) {};
  \node[right=1 of A] (d2) {};
  \node[node, right=1 of d2] (D) {};
  \node[node] (C) at (d1-|d2) {};
  \node[node, above=0.5 of d2] (F) {};

  \draw[edge, p2] (A) to (C);
  \draw[edge, p2, dotted] (C) to (D);
  \draw[edge, p2, dotted] (A) to (F);
  \draw[edge, p2] (D) to (F);
\end{tikzpicture}
}{\begin{tikzpicture}
  \begin{scope}[on background layer]
    \node[node] (A) {};
    \node[below=0.5 of A] (d1) {};
    \node[right=1 of A] (d2) {};
    \node[node, right=1 of d2] (D) {};
    \node[node] (C) at (d1-|d2) {};
    \node[node, above=0.5 of d2] (F) {};

    \draw[edge, p2] (A) to (C);
    \draw[edge, p2, dotted] (C) to (D);
    \draw[edge, p2, dotted] (A) to (F);
    \draw[edge, p2] (D) to (F);
  \end{scope}
  \node[fill=green!25, circle, draw, text=black, inner sep=2pt] (value) at (d2) {\huge$2$};
\end{tikzpicture}
}}
      \resizebox{\cutscaling}{!}{\alt<1>{\begin{tikzpicture}
  \node[node] (A) {};
  \node[below=0.5 of A] (d1) {};
  \node[right=1 of A] (d2) {};
  \node[node, right=1 of d2] (D) {};
  \node[node] (C) at (d1-|d2) {};
  \node[node, above=0.5 of d2] (F) {};

  \draw[edge, p2] (A) to (C);
  \draw[edge, p2] (C) to (D);
  \draw[edge, p2, dotted] (A) to (F);
  \draw[edge, p2, dotted] (D) to (F);
\end{tikzpicture}
}{\begin{tikzpicture}
  \node[node] (A) {};
  \node[below=0.5 of A] (d1) {};
  \node[right=1 of A] (d2) {};
  \node[node, right=1 of d2] (D) {};
  \node[node] (C) at (d1-|d2) {};
  \node[node, above=0.5 of d2] (F) {};

  \node[above left=0.25 and 0.25 of d2] (cross1) {};
  \node[above right=0.25 and 0.25 of d2] (cross2) {};
  \node[below right=0.25 and 0.25 of d2] (cross3) {};
  \node[below left=0.25 and 0.25 of d2] (cross4) {};

  \draw[very thick, red] (cross1) to (cross3);
  \draw[very thick, red] (cross2) to (cross4);

  \draw[edge, p2] (A) to (C);
  \draw[edge, p2] (C) to (D);
  \draw[edge, p2, dotted] (A) to (F);
  \draw[edge, p2, dotted] (D) to (F);
\end{tikzpicture}
}}
      \column{0.5\textwidth}
      \resizebox{\cutscaling}{!}{\alt<1>{\begin{tikzpicture}
  \node[node] (A) {};
  \node[below=0.5 of A] (d1) {};
  \node[right=1 of A] (d2) {};
  \node[node, right=1 of d2] (D) {};
  \node[node] (C) at (d1-|d2) {};
  \node[node, above=0.5 of d2] (F) {};

  \draw[edge, p2, dotted] (A) to (C);
  \draw[edge, p2] (C) to (D);
  \draw[edge, p2] (A) to (F);
  \draw[edge, p2, dotted] (D) to (F);
\end{tikzpicture}
}{\begin{tikzpicture}
  \node[node] (A) {};
  \node[below=0.5 of A] (d1) {};
  \node[right=1 of A] (d2) {};
  \node[node, right=1 of d2] (D) {};
  \node[node] (C) at (d1-|d2) {};
  \node[node, above=0.5 of d2] (F) {};

  \node[above left=0.25 and 0.25 of d2] (cross0.5) {};
  \node[above right=0.25 and 0.25 of d2] (cross2) {};
  \node[below right=0.25 and 0.25 of d2] (cross3) {};
  \node[below left=0.25 and 0.25 of d2] (cross4) {};

  \draw[very thick, red] (cross1) to (cross3);
  \draw[very thick, red] (cross2) to (cross4);

  \draw[edge, p2, dotted] (A) to (C);
  \draw[edge, p2] (C) to (D);
  \draw[edge, p2] (A) to (F);
  \draw[edge, p2, dotted] (D) to (F);
\end{tikzpicture}
}}
      \resizebox{\cutscaling}{!}{\alt<1>{\begin{tikzpicture}
  \node[node] (A) {};
  \node[below=0.5 of A] (d1) {};
  \node[right=1 of A] (d2) {};
  \node[node, right=1 of d2] (D) {};
  \node[node] (C) at (d1-|d2) {};
  \node[node, above=0.5 of d2] (F) {};

  \draw[edge, p2, dotted] (A) to (C);
  \draw[edge, p2] (C) to (D);
  \draw[edge, p2, dotted] (A) to (F);
  \draw[edge, p2] (D) to (F);
\end{tikzpicture}
}{\begin{tikzpicture}
  \node[node] (A) {};
  \node[below=0.5 of A] (d1) {};
  \node[right=1 of A] (d2) {};
  \node[node, right=1 of d2] (D) {};
  \node[node] (C) at (d1-|d2) {};
  \node[node, above=0.5 of d2] (F) {};

  \node[above left=0.25 and 0.25 of d2] (cross1) {};
  \node[above right=0.25 and 0.25 of d2] (cross2) {};
  \node[below right=0.25 and 0.25 of d2] (cross3) {};
  \node[below left=0.25 and 0.25 of d2] (cross4) {};

  \draw[very thick, red] (cross1) to (cross3);
  \draw[very thick, red] (cross2) to (cross4);

  \draw[edge, p2, dotted] (A) to (C);
  \draw[edge, p2] (C) to (D);
  \draw[edge, p2, dotted] (A) to (F);
  \draw[edge, p2] (D) to (F);
\end{tikzpicture}
}}
      \resizebox{\cutscaling}{!}{\alt<1-3>{\begin{tikzpicture}
  \node[node] (A) {};
  \node[below=0.5 of A] (d1) {};
  \node[right=1 of A] (d2) {};
  \node[node, right=1 of d2] (D) {};
  \node[node] (C) at (d1-|d2) {};
  \node[node, above=0.5 of d2] (F) {};

  \draw[edge, p2] (A) to (C);
  \draw[edge, p2, dotted] (C) to (D);
  \draw[edge, p2] (A) to (F);
  \draw[edge, p2, dotted] (D) to (F);
\end{tikzpicture}
}{\begin{tikzpicture}
  \begin{scope}[on background layer]
    \node[node] (A) {};
    \node[below=0.5 of A] (d1) {};
    \node[right=1 of A] (d2) {};
    \node[node, right=1 of d2] (D) {};
    \node[node] (C) at (d1-|d2) {};
    \node[node, above=0.5 of d2] (F) {};

    \draw[edge, p2] (A) to (C);
    \draw[edge, p2, dotted] (C) to (D);
    \draw[edge, p2] (A) to (F);
    \draw[edge, p2, dotted] (D) to (F);
  \end{scope}
  \node[fill=green!25, circle, draw, text=black, inner sep=2pt] (value) at (d2) {\huge$5$};
\end{tikzpicture}
}}
    \end{columns}
  \end{columns}
  \uncover<9>{ }
\end{frame}

\begin{frame}
  \frametitle{Soundness}
  \begin{itemize}
    \item Termination:
      \begin{itemize}
        \item<2-> inductively applying contraction operations lead to basis
        \item<3-> deletion cannot \enquote{loose} bases for other players since
          chosen cut is inclusion-wise minimal
      \end{itemize}
    \item Correctness
      \begin{itemize}
        \item<4-> Lemma: cost of the chosen element decreases monotonically in each
          iteration
        \item<5-> assume there is a beneficial alternative basis
          $\hat{B} = (B_{i}\setminus\set{g})\cup{f}$
          \begin{itemize}
            \item<6-> consider iteration $k$ in which $g$ is added to basis of $i$:
            \item<7-> if $f$ is in the corresonponding cut its cost never changes
              since it is deleted from every matroid $\rightarrow$ $f$ is not
              in the cut
            \item<8-> but $f$ and $g$ \enquote{open} same dimensionality, thus $f$
              is not left in contracted matroid of $i$
            \item<9-> $f$ was deleted before (in this iteration the cost of the
              chosen element was lower than for $f$ in contradiction to lemma)
          \end{itemize}
      \end{itemize}
  \end{itemize}
\end{frame}

\section{Weighted Matroid Games}
\begin{frame}
  \frametitle{Recall: Congestion Model}
  \begin{equation*}
    \mathcal{M} = \tupel{N, E, \mathcal{B}, \left(d_{i}\right)_{i\in N},
    \left(c_{e}\right)_{e\in E}}
  \end{equation*}
  \vspace{-1cm}
  \begin{itemize}
    \item before: $d_{i} = 1$ for all $i$
    \item<2-> now: $d_{i}\in\mathbb{N}_{>0}$ 
    \item<3-> auxiliary definitions:
      \begin{itemize}
        \item<4-> $N_{e}\tupel{B} = \set{i\in N\middle|e\in B_{i}}$
        \item<5-> for a fixed $B$: $\ex_{i}\tupel{e} =
          \set{f\in E\setminus\set{e}\middle|
          (B_{i}\setminus\set{e})\cup\set{f}\in\mathcal{B}_{i}}$: set of resources
          that can replace $e$ for player $i$
        \item<6-> $F = \bigcup_{i\in N}\bigcap_{B\in \mathcal{B}_{i}} B$:
          set of obligatory elements for at least one player
        \item<7-> $\Delta_{i}\tupel{B;e\rightarrow f} =
          c_{f}\tupel{\ell_{f}\tupel{B_{i}\cup\set{f}\setminus\set{e}}} -
          c_{f}\tupel{\ell_{f}\tupel{B}}$: marginal cost to buy resource $f$ if
          $i$ switches
        \item<8-> $\Delta_{i}^{e} = \min_{f\in\ex_{i}(e)}\set{\Delta_{i}(B;e\rightarrow f)}$:
          minimal marginal cost for exchanging $e$
      \end{itemize}
  \end{itemize}
\end{frame}

\begin{frame}
  \frametitle{Characterisation of Pure Nash Equilibria}
  \begin{theorem}
    For a weighted matroid resource buying game
    $\tupel{N, E, \mathcal{B}, d, c}$ and $B\in\mathcal{B}$ the following
    holds: there is a payment vector $p$ such that $\tupel{B,p}$ is a
    PNE if and only if
    \begin{equation*}
      c_{e}(B) \leq \sum_{i\in N_{e}(B)}\Delta_{i}^{e}(B)
      \textit{ for all }e\in E\setminus F
    \end{equation*}
  \end{theorem}
  \begin{proof}
    if: \uncover<2->{$p_{i}^{e} =
      \begin{cases}
        \frac{\Delta_{i}^{e}(B)}{\sum_{j\in N_{e}(B)}\Delta_{j}^{e}(B)}\cdot
        c_{e}(B)&\text{ if } \sum_{j\in N_{e}(B)}\Delta_{j}^{e}(B) > 0\\
        0 &\text{ otherwise}
      \end{cases}$\\
      for all $e\in E\setminus F$ and $i\in N_{e}(B)$\\}
    only if: \uncover<3->{for all $i$ and $e$: $p_{i}^{e} \leq \Delta_{i}^{e}(B)$
      and $c_{e}(B) = \sum_{i\in N_{e}(B)}p_{i}^{e}$}
  \end{proof}
\end{frame}

\begin{frame}
  \frametitle{Existence of Pure Nash Equilibria}
  \begin{theorem}
    Every weighted matroid resource buying game with marginally non-increasing
    cost functions possesses a PNE.
  \end{theorem}
  \begin{itemize}
    \item<2-> socially optimal configuration profile: $B\in\mathcal{B}$ such that
      $\sum_{e\in E}c_{e}(\ell_{e}(B))$ is minimal
    \item<3-> for socially optimal configuration profile $B$ holds:
      \begin{equation*}
        c_{e}(B)\leq \sum_{i\in N_{e}(B)}\Delta_{i}^{e}(B)
        \text{ for all }e\in E\setminus F
      \end{equation*}
    \item<4-> for $B$ exists $p$ s.t. $\tupel{B,p}$ is a PNE
    \item<5-> (but finding socially optimal configuration profiles is NP-hard)
  \end{itemize}
\end{frame}

\section{Non-Matroid Strategy Spaces}
\begin{frame}
\frametitle{Consequences of using Non-Matroid Strategy Spaces}
  \begin{lemma}
    If $\mathcal{S}\subseteq 2^{E}$ is a non-matroid anti-chain, then there
    exist $X,Y\in\mathcal{S}$ and $\set{a,b,c}\subseteq X\cup Y$ such that each
    set $Z\in\mathcal{S}$ with $Z\subseteq(X\cup Y)\setminus\set{a}$ contains
    $b$ and $c$. (Especially, there is $S\in\mathcal{S}$ with
    $S\subseteq X\cup Y$ and $a\in S$ but $b,c\notin S$.)
  \end{lemma}
  \uncover<2->{\begin{theorem}
    For every non-matroid anti-chain $\mathcal{S}$ on a set of resource $E$,
    there exists a weighted two-player resource buying game
    $G = \tupel{\tilde{E}, \tupel{\mathcal{S}_{1} \times
    \mathcal{S}_{2}}\times \mathcal{P}, \pi}$ having marginally non-increasing
    cost functions, whose strategy spaces $\mathcal{S}_{1}$ and
    $\mathcal{S}_{2}$ are both isomophic to $\mathcal{S}$, so that $G$ does not
    possess a pure Nash equilibrium.
  \end{theorem}}
\end{frame}

\begin{frame}
  \frametitle{Construction of games without PNE}
   \begin{columns}
     \column{0.3\textwidth}
       \begin{itemize}
         \item $c_{x}(t) = t$
         \item $c_{y}(t) = 5\frac{1}{2}$
         \item $c_{z}(t) = 4$
         \item $d_{1} = 5$
         \item $d_{2} = 4$
       \end{itemize}
     \column{0.7\textwidth}
       \begin{center}
         \resizebox{\textwidth}{!}{\begin{tikzpicture}
  \node (shared) {
    $\begin{aligned}
      a_1 = x = b_2\\
      b_1 = y = a_2\\
      c_1 = z = c_2
    \end{aligned}$
  };

  \node (p3) at (shared.south east) {};
  \node[above=3 of p3] (p2) {};
  \node[left=4.5 of p3] (m2) {};
  \node[left=1 of m2] (p4) {};
  \node (m1) at (p2-|m2) {};
  \node (p1) at (m1-|p4) {};
  \node [left=0.4 of shared.north west] (text1) {$X_{1}\cup Y_{1}$};
  \fill[gray!50] (p1.center) -- (m1.center) -- (m2.center) -- (p4.center) -- cycle;
  \node [below left=0.5 and 0.5 of p4] (mark1) {$M$};
  \node [below=0.5 of m2] (mark3) {$0$};
  \node [above right=0.1 and 1.7 of m1] (S1) {$S_{1}$};
  
  \node (p5) at (shared.north west) {};
  \node[right=4.5 of p5] (m3) {};
  \node[right=1 of m3] (p6) {};
  \node[below=3 of p5] (p8) {};
  \node (m4) at (p8-|m3) {};
  \node (p7) at (m4-|p6) {};
  \node [right=0.4 of shared.south east] (text2) {$X_{2}\cup Y_{2}$};
  \fill[gray!50] (p6.center) -- (m3.center) -- (m4.center) -- (p7.center) -- cycle;
  \node [above right=0.5 and 0.5 of p6] (mark2) {$M$};
  \node [above=0.5 of m3] (mark4) {$0$};
  \node [above left=0.1 and 0.4 of m3] (S2) {$S_{2}$};

  \draw[p1] (p1.center) to (m1.center);
  \draw[p1] (m1.center) to (p2.center);
  \draw[p1] (p2.center) to (p3.center);
  \draw[p1] (p3.center) to (m2.center);
  \draw[p1] (m2.center) to (p4.center);
  \draw[p1] (p4.center) to (p1.center);
  \draw[dashed, p1] (m1.center) to (m2.center);
  \draw (mark1) to (p4.north east);
  \draw (mark3) to (text1.south);

  \draw[p2] (p5.center) to (m3.center);
  \draw[p2] (m3.center) to (p6.center);
  \draw[p2] (p6.center) to (p7.center);
  \draw[p2] (p7.center) to (m4.center);
  \draw[p2] (m4.center) to (p8.center);
  \draw[p2] (p8.center) to (p5.center);
  \draw[dashed, p2] (m3.center) to (m4.center);
  \draw (mark2) to (p6.south west);
  \draw (mark4) to (text2.north);
\end{tikzpicture}
}
       \end{center}
   \end{columns}
   \begin{itemize}
     \item<2-> assume PNE $\tupel{Z^{\ast} = \tupel{Z_{1}^{\ast}, Z_{2}^{\ast}},
       p^{\ast} = \tupel{p_{1}^{\ast}, p_{2}^{\ast}}}$
     \item<3-> $Z_{i}^{\ast}\subseteq\tupel{X_{i}\cup Y_{i}}$
     \item<4-> $x\not\in Z_{1}^{\ast}\Rightarrow \set{y,z}\subseteq Z_{1}^{\ast}$
     \item<5-> $y\not\in Z_{2}^{\ast}\Rightarrow \set{x,z}\subseteq Z_{2}^{\ast}$
     \item<6->[$\Rightarrow$] neiter $x\in Z_{1}^{\ast}$ nor $x\not\in Z_{1}^{\ast}$ $\lightning$
   \end{itemize}
\end{frame}

\begin{frame}
  \frametitle{Non-Decreasing Marginal Cost Functions}
  \begin{columns}
    \column{0.5\textwidth}
      Non-Increasing Marginal Cost Functions:
      \begin{center}
        \begin{tikzpicture}
          \begin{axis}[
              xlabel=$x$,
              ylabel=$c_{e}(x)$,
              smooth,
              width=5cm,
              height=3cm,
              ticks=none,
              no markers
            ]
              \addplot{ln(x)};
          \end{axis}
        \end{tikzpicture}
      \end{center}
    \column{0.5\textwidth}
      Non-Decreasing Marginal Cost Functions:
      \begin{center}
        \begin{tikzpicture}
          \begin{axis}[
              xlabel=$x$,
              xmin=1,
              xmax=4,
              ylabel=$c_{e}(x)$,
              smooth,
              width=5cm,
              height=3cm,
              ticks=none,
              no markers
            ]
              \addplot{5*exp(x)};
          \end{axis}
        \end{tikzpicture}
      \end{center}
  \end{columns}
  \uncover<2->{\begin{theorem}
    Let $G$ be a weighted resource buying game with non-decreasing marginal
    costs and let $S^{\ast}$ be a socially optimal solution. We can compute
    (so called) marginal cost pricing vectors which induce a PNE.
  \end{theorem}}
\end{frame}

\section{Conclusion}
\begin{frame}
  \frametitle{Results}
  \begin{itemize}
    \item Resource Buying Games
    \item<2-> Matroids:
      \parbox{6cm}{
        \begin{itemize}
          \item Operations: Contraction, Deletion
          \item Cuts
        \end{itemize}
      }
    \item<3-> Games:
      \parbox{8cm}{
        \begin{itemize}
          \item \parbox{6cm}{
              \begin{tabular}{|l|ll|}
                \hline
                Non-increasing marginal & Matroid & No Matroid\\\hline
                unweighted & Algorithm & \\
                weighted & PNE & No PNE\\\hline
                Non-decreasing marginal & PNE & \\\hline
              \end{tabular}
            }
        \end{itemize}
      }
    \item<4-> Slides: \url{github.com/weltoph/sem-alg-gametheory}
  \end{itemize}
  \uncover<5->{\centering Thank you for your attention!}
\end{frame}

\end{document}
