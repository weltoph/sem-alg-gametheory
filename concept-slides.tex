\documentclass{beamer}
\usetheme{metropolis}
\title{Resource Buying Games}
\author[C. Welzel]{Christoph Welzel}

\addtobeamertemplate{frametitle}{\let\insertframetitle\insertsubsectionhead}{}
\newcommand{\ft}{\frametitle{dummy}}

\begin{document}
\maketitle
\section{Introduction}
\begin{frame}
  \begin{itemize}
    \item Context
      \begin{itemize}
        \item What is modelled?
        \item How is it modelled?
      \end{itemize}
    \item Example
      \begin{itemize}
        \item Introducing an example which is used to illustrate different
          aspects throughout the presentation
        \item Interleaved graphs for two players, both want to acquire a
          spanning tree
      \end{itemize}
  \end{itemize}
\end{frame}

\subsection{Game definition}
\begin{frame}
  \ft
  \begin{itemize}
    \item Formally define game
    \item Refer to the example
  \end{itemize}
\end{frame}
\subsection{Pure Nash equilibria}
\begin{frame}
  \ft
  \begin{itemize}
    \item Explain what PNEs are (maybe from previous presentations this
      knowledge can be assumed)
    \item Again the example can be used to illustrate what uniliteral change
      might imply
  \end{itemize}
\end{frame}
\subsection{Basic matroid theory}
\begin{frame}
  \ft
  \begin{itemize}
    \item Formally define matroids
    \item Explain the meaning of the definition by using the example
    \item Contraction and deletion (again with example)
    \item Define Matroid games (and mention that the example fits)
  \end{itemize}
\end{frame}

\section{Algorithm: PNE in Unweighted Matroid Games}
\subsection{Algorithm}
\begin{frame}
  \ft
  \begin{itemize}
    \item Present algorithm
    \item Execute one or two steps at example
  \end{itemize}
\end{frame}

\subsection{Correctness}
\begin{frame}
  \ft
  \begin{itemize}
    \item Intuitively present the ideas for the correctness
    \item (Formal proofs might take too long?)
  \end{itemize}
\end{frame}

\section{Weighted Matroid Games}
\begin{frame}
  \begin{itemize}
    \item Introduce socially optimal solution
    \item Show (or at least motivate) why this is a PNE
    \item Mention reduction to show that finding an optimal solution is NP-hard
  \end{itemize}
\end{frame}

\section{Non-Matroid Strategy Spaces}
\subsection{Lack of PNE}
\begin{frame}
  \ft
  \begin{itemize}
    \item Present that for Non-Matroid Games we can find strategy spaces
      which do not allow PNE
    \item Illustrate the construction by a picture
  \end{itemize}
\end{frame}

\subsection{Non-Decreasing Marginal Cost Functions \& Price of Anarchy}
\begin{frame}
  \ft
  \begin{itemize}
    \item Present that games with Non-Decreasing Marginal Cost Functions allow
      PNE (explain how the marginal cost pricing is obtained)
    \item Define Price of Anarchy
    \item Mention that non-increasing and non-decreasing games with weighted
      players is unbounded
  \end{itemize}
\end{frame}

\section{Conclusion}
\begin{frame}
  \begin{itemize}
    \item Sum up results
    \item Embed results into topic
  \end{itemize}
\end{frame}

\end{document}
